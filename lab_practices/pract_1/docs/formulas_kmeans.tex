\documentclass[a4paper,12pt]{article}

\usepackage{amsmath}

\begin{document}

\pagestyle{empty}

Let $\mathcal{C}_{k}$ be one of the target classes of the task, in this use case
one of the possible states the patient can be while he/she is monitored,
and let $x_{t}$ be the observed sample at time instant $t$,
then, for this use case, $P(\mathcal{C}_{k} \mid x_{t})$ represents the
\emph{a posteriori} probability that the patient is in the state corresponding
to target class $\mathcal{C}_{k}$ when the sample $x_{t}$ has been observed.

\bigskip

Applying the Bayes' rule, considering each channel $l$ as independent of the others (a Naive assumption)
and expanding the joint probabilities to take into account the clustering, we get the following 
expression:

\[
    P(\mathcal{C}_{k} \mid x_{t}) = 
    \frac{p(\mathcal{C}_{k}, x_{t})}{\underset{i=1}{\overset{K}{\sum}} p(\mathcal{C}_{i}, x_{t})} =
    \frac{\underset{l=1}{\overset{L}{\sum}} p(\mathcal{C}_{k}, x_{t}^{l})}{\underset{i=1}{\overset{K}{\sum}} \; \underset{l=1}{\overset{L}{\sum}} p(\mathcal{C}_{i}, x_{t}^{l})} =
    \frac{\underset{l=1}{\overset{L}{\sum}} \; \underset{j=1}{\overset{J}{\sum}} p(\mathcal{C}_{k}, w_{j}, x_{t}^{l})}{\underset{i=1}{\overset{K}{\sum}} \; \underset{l=1}{\overset{L}{\sum}} \; \underset{j=1}{\overset{J}{\sum}} p(\mathcal{C}_{i}, w_{j}, x_{t}^{l})}
\]
where
\begin{center}
\begin{tabular}{|r|p{100mm}|}
\hline
$K$ & is the number of target classes \\
\hline
$J$ & is the number of clusters in the clustering, i.e., the number of mean vectors in the codebook \\
\hline
$L$ & is the number of channels recorded in the EEG \\
\hline
$x_{t}$ & is the sample at time $t$ including all the channels in the EGG \\
\hline
$x_{t}^{l}$ & is the sample at time $t$ corresponding to the $l$-th channel of the EGG \\
\hline
$p(\mathcal{C}_{k}, x_{t})$ & is the joint probability of target class $\mathcal{C}_{k}$ and sample $x_{t}$ \\
\hline
$p(\mathcal{C}_{k}, x_{t}^{l})$ & is the joint probability of target class $\mathcal{C}_{k}$ and sample $x_{t}^{l}$ \\
\hline
$p(\mathcal{C}_{k}, w_{j}, x_{t}^{l})$ & is the joint probability of target class $\mathcal{C}_{k}$, cluster $w_{j}$ and sample $x_{t}^{l}$ \\
\hline
\end{tabular}
\end{center}

\newpage

In more detail
\[
p(\mathcal{C}_{k}, w_{j}, x_{t}^{l}) =
p(x_{t}^{l} \mid w_{j}, \mathcal{C}_{k}) \cdot P(w_{j} \mid \mathcal{C}_{k}) \cdot P(\mathcal{C}_{k})
\]
where
\begin{center}
\begin{tabular}{rp{110mm}}
$p(x_{t}^{l} \mid w_{j}, \mathcal{C}_{k})$ & is the conditional probability density of observing sample
                                             $x_{t}^{l}$ when the patient is in the state corresponding
                                             to target class $\mathcal{C}_{k}$ and the generated samples
                                             in such state fall mainly in cluster $w_j$,
                                             this will be approached by $p(x_{t}^{l} \mid w_{j})$, \\
$p(x_{t}^{l} \mid w_{j})$ & is the conditional probability density that observing sample $x_{t}^{l}$
                            belongs to cluster $w_{j}$, that is computed simply by
                            $$p(x_{t}^{l} \mid w_{j}) = \Bigl\{ \begin{array}{l} 1 \; \texttt{if} \; x_{t}^{l} \; \texttt{falls in cluster} \; w_{j} \\ 0 \; \texttt{otherwise} \\ \end{array}$$ \\

$P(w_{j} \mid \mathcal{C}_{k})$ & is the conditional probability of observing samples falling in cluster $w_{j}$ when
                                  the patient is in the state corresponding to target class $\mathcal{C}_{k}$, computed
                                  as $$P(w_{j} \mid \mathcal{C}_{k}) = \frac{count(w_{j}, \mathcal{C}_{k})}{\sum_{h=1}^{J} count(w_{h}, \mathcal{C}_{k})}$$ \\

$P(\mathcal{C}_{k})$ & is the \emph{a priori} probablity of target class $\mathcal{C}_{k}$, computed as 
                                  as $$P(\mathcal{C}_{k}) = \frac{count(\mathcal{C}_{k})}{\sum_{i=1}^{K} count(\mathcal{C}_{k})}$$ \\
\end{tabular}
\end{center}

Finally,
\[
P(\mathcal{C}_{k} \mid x_{t}) \approx
    \frac{\underset{l=1}{\overset{L}{\sum}} \; \underset{j=1}{\overset{J}{\sum}}
        p(x_{t}^{l} \mid w_{j}) \cdot P(w_{j} \mid \mathcal{C}_{k}) \cdot P(\mathcal{C}_{k})}%
        {\underset{i=1}{\overset{K}{\sum}} \; \underset{l=1}{\overset{L}{\sum}} \; \underset{j=1}{\overset{J}{\sum}}
            p(x_{t}^{l} \mid w_{j}) \cdot P(w_{j} \mid \mathcal{C}_{i}) \cdot P(\mathcal{C}_{i})}
\]
\end{document}
