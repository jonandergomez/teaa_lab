\documentclass[a4paper,12pt]{article}

\usepackage{amsmath}
\usepackage{xcolor}
\usepackage[left=15mm, right=15mm]{geometry}

%\setlength{\textwidth}{170mm}

\begin{document}

\pagestyle{empty}

Let $\mathcal{C}_{k}$ be one of the target classes of the task, in this use case
one of the possible states the patient can be while he/she is monitored,
and let $x_{t}$ be the observed sample at time instant $t$,
then, for this use case, $P(\mathcal{C}_{k} \mid x_{t})$ represents the
\emph{a posteriori} probability that the patient is in the state corresponding
to target class $\mathcal{C}_{k}$ when the sample $x_{t}$ has been observed.

\bigskip

Applying the Bayes' rule, considering each channel $l$ as independent of the others (a Naive assumption)
and expanding the joint probabilities to take into account the clustering, we get the following 
expression when the clustering is based on a Gaussian Mixture Model (GMM):



\[
P(\mathcal{C}_{k} \mid x_{t})
    = \frac{\pi(\mathcal{C}_{k}) \cdot p(x_{t} \mid \mathcal{C}_{k})}{\underset{c=1}{\overset{K}{\sum}} \;  \pi(\mathcal{C}_{c}) \cdot p(x_{t} \mid \mathcal{C}_{c})}
    = \frac{\pi(\mathcal{C}_{k}) \cdot \underset{l=1}{\overset{L}{\sum}} \; p(x_{t}^{l} \mid \mathcal{C}_{k})}{\underset{c=1}{\overset{K}{\sum}} \; \pi(\mathcal{C}_{c}) \cdot \underset{l=1}{\overset{L}{\sum}} \; p(x_{t}^{l} \mid \mathcal{C}_{c})}
\]
where
\begin{center}
\begin{tabular}{|r|p{130mm}|}
\hline
$K$ & is the number of target classes \\
\hline
$L$ & is the number of channels recorded in the EEG \\
\hline
$x_{t}$ & is the sample at time $t$ including all the channels in the EGG \\
\hline
$x_{t}^{l}$ & is the sample at time $t$ corresponding to the $l$-th channel of the EGG \\
\hline
$\pi(\mathcal{C}_{k})$ & is the \emph{a priori} probability of the target class $\mathcal{C}_{k}$ \\
\hline
$p(x_{t}^{l} \mid \mathcal{C}_{k})$ & is the \emph{conditional probability density}
                                    of generating $x_{t}^{l}$ during the period in which the state
                                    of the patient corresponds to target class $\mathcal{C}_{k}$ \\
\hline
\end{tabular}
\end{center}

\medskip

The \emph{conditional probability density} $p(x_{t}^{l} \mid \mathcal{C}_{k})$ is computed as follows:

\[
p(x_{t}^{l} \mid \mathcal{C}_{k}) = \sum_{j=1}^J \; \pi(w_j) \cdot {\color{red} p(x_{t}^{l} \mid w_j, \mathcal{C}_{k})} \cdot P(w_j \mid \mathcal{C}_{k})
\simeq \sum_{j=1}^J \; \pi(w_j) \cdot {\color{red}p(x_{t}^{l} \mid w_j)} \cdot P(w_j \mid \mathcal{C}_{k})
\]
where
\begin{center}
\begin{tabular}{|r|p{130mm}|}
\hline
$J$ & is the number of clusters in the clustering, i.e., the number of components of the GMM in use \\
\hline
$\pi(w_j)$ & is the weight of Gaussian $w_j$ in the mixture \\
\hline
$p(x_{t}^{l} \mid w_j)$ & is computed as a normal probability distribution $\mathcal{N}(x_{t}^{l} \mid \mu_{j}, \Sigma_{j})$ \\
\hline
$P(w_j \mid \mathcal{C}_{k})$ & is the conditional probability of observing samples generated by the
                              Gaussian component $w_j$ during a period corresponding
                              to target class $\mathcal{C}_{k}$ \\
\hline
\end{tabular}
\end{center}
then, 
\[
P(\mathcal{C}_{k} \mid x_{t}) \simeq
    \frac{\pi(\mathcal{C}_{k}) \cdot \underset{l=1}{\overset{L}{\sum}} \; \underset{j=1}{\overset{J}{\sum}} \; \pi(w_{j}) \cdot p(x_{t}^{l} \mid w_{j}) \cdot P(w_{j} \mid \mathcal{C}_{k})}%
           {\underset{c=1}{\overset{K}{\sum}} \; \pi(\mathcal{C}_{c}) \cdot \underset{l=1}{\overset{L}{\sum}} \; \underset{j=1}{\overset{J}{\sum}} \; \pi(w_{j}) \cdot p(x_{t}^{l} \mid w_{j}) \cdot P(w_{j} \mid \mathcal{C}_{c})}
\]

\end{document}
