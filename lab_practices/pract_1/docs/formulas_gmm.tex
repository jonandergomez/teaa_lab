\documentclass[a4paper,12pt]{article}

\usepackage{amsmath}

\setlength{\textwidth}{170mm}

\begin{document}

\pagestyle{empty}

As in the previous case, the confusion matrix must be normalized, but in this case to contain the
conditional probabilities $P(w_j \mid \mathcal{C}_k)$,
where $w_j$ is the $j$-th component of the GMM in use and $\mathcal{C}_k$ is the $k$-th target class.

This way, $P(\mathcal{C}_k \mid x)$, the \emph{a posteriori} probability that
the current period corresponds to target class $\mathcal{C}_k$ when $x$ has been observed,
will be computed as follows according to the Bayes' rule:
\[
P(\mathcal{C}_k \mid x) = \frac{\pi(\mathcal{C}_k) \cdot p(x \mid \mathcal{C}_k)}{\sum_c \pi(\mathcal{C}_c) \cdot p(x \mid \mathcal{C}_c)}
\]
where
\begin{itemize}
\item
$\pi(\mathcal{C}_k)$ is the \emph{a priori} probability of the target class $\mathcal{C}_k$,
and

\item
$p(x \mid \mathcal{C}_k)$ is the \emph{conditional probability density}
of generating $x$ during the period in which the state of the patient
corresponds to target class $\mathcal{C}_k$.
\end{itemize}

\bigskip

The \emph{conditional probability density} $P(x \mid \mathcal{C}_k)$ is computed as follows:

\[
p(x \mid \mathcal{C}_k) = \sum_{j=1}^J \pi(w_j) \cdot p(x \mid w_j) \cdot P(w_j \mid \mathcal{C}_k)
\]
where
\begin{itemize}
\item
$J$ is the number of components in the GMM in use,

\item
$\pi(w_j)$ is the weight of Gaussian $w_j$ in the mixture,

\item
$p(x \mid w_j) = \mathcal{N}(x \mid \mu_{j}, \Sigma_{j})$ and

\item
$P(w_j \mid \mathcal{C}_k)$ is the conditional probability of observing
samples generated by the Gaussian component $w_j$ during a period corresponding
to target class $\mathcal{C}_k$.
\end{itemize}

\end{document}
